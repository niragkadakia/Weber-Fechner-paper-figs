%!TEX root = main.tex

Here is a citation to a seminal work\cite{OKeeNade78a}. \lipsum[2-3]

\section*{Results}

\subsection*{A model framework that preserves ORN tuning, diversity, and adaptive feedback}

We propose a two-step model framework of odor sensing and discrimination. At the first stage, odor molecules individually bind and unbind to distinct olfactory receptor neurons (ORNs), which reside in either an active (firing) or inactive (quiescent) state. When bombarded by a naturalistic odor consisting of several volatile compounds, the likelihood that an ORN is active is dictated by two aspects. First, the probability it binds a constituent molecular odorant depends on both ORN and odorant identity. Second, the likelihood that this binding event then causes an action potential depends on ORN-specific activation energies required to open Na and Ca ion channels. 

We first show that the natural diversity of observed tuning curve response in \textit{Drosophila} response is 




The affinities of distinct molecules to distinct ORNs are completely independent and general. 

\subsection*{}



%An inline figure reference will look like \tfig{single}, whereas a parenthesized figure panel
%reference will be abbreviated (\fig{single}{a}). \comment{Remove `disable' from the todonotes
%package arguments to display helpful inline comments like these. Additional commands can be defined
%for different authors.} \lipsum[4]

\subsection*{The second major result}

Equations can be referenced such as \eqn{skaggs} and other papers as well\cite{Hill78a}. References
that only appear in the supplementary materials (figure captions, etc.) or online methods section
will follow the reference numbering of the main text and appear at the end of the main reference
list. \lipsum[5]

Important results in the supplementary data (\suppfig{template}). \lipsum[6]

\section*{Discussion}

\lipsum[7-9]

