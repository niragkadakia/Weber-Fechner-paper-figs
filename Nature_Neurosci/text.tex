%!TEX root = main.tex

Here is a citation to a seminal work\cite{OKeeNade78a}. \lipsum[2-3]

\section*{Results}

\subsection*{A model framework that preserves ORN tuning, diversity, and adaptive feedback}

The theoretical framework of odor discrimination consists of two stages: a biophysical model of odor binding and subsequent ORN firing -- "encoding" -- and a computational paradigm for then inferring odor identity and intensity from this repertoire of ORN response, or "decoding." We model the first stage as follows. Odor molecules individually bind and unbind to distinct olfactory receptor neurons (ORNs), which are either active (firing) or inactive (quiescent) state. When bombarded by a naturalistic odor consisting of a few or more volatile compounds, the likelihood and rate of ORN spiking is dictated by two aspects. First, the probability that ORN binds a given molecular odorant depends on the identities of both the neuron and the odorant, as well as the neuron's state (active or inactive). Second, the likelihood that this binding event then incites an action potential depends on ORN-specific activation energies needed to open Na and Ca ion channels. Together, these comprise a stochastic process that translate odors of varying identities and concentrations into a repertoire of ORN response.

To characterize the diversity of ORN response, w

At the second stage, the ORN activity levels is incorporated into an optimization routine intended to mimic the neural computations that infer the identity and intensity of the odor signal. This is the decoding aspect, of which 



We first show that the natural diversity of observed tuning curve response in \textit{Drosophila} response is 




The affinities of distinct molecules to distinct ORNs are completely independent and general. 

\subsection*{}



%An inline figure reference will look like \tfig{single}, whereas a parenthesized figure panel
%reference will be abbreviated (\fig{single}{a}). \comment{Remove `disable' from the todonotes
%package arguments to display helpful inline comments like these. Additional commands can be defined
%for different authors.} \lipsum[4]

\subsection*{The second major result}

Equations can be referenced such as \eqn{skaggs} and other papers as well\cite{Hill78a}. References
that only appear in the supplementary materials (figure captions, etc.) or online methods section
will follow the reference numbering of the main text and appear at the end of the main reference
list. \lipsum[5]

Important results in the supplementary data (\suppfig{template}). \lipsum[6]

\section*{Discussion}

\lipsum[7-9]

