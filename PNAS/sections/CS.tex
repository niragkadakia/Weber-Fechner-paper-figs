%% CS description in main text

\subsection*{Decoding sparse signals}
It has been observed that despite the vast array of distinct odors between which humans and other animals can discriminate, individual odors found in the natural environment contain only a handful of monomolecular components -- odor signals are \textit{sparse} in the $N$-dimensional space of constituent volatile molecules. Even without other mechanisms such as temporal dynamics, the theory of compressed sensing might explain how so many odors are reliably discriminated by a few dozen ORNs. In compressed sensing, signal sparsity acts as a regulator that renders the inverse problem well-posed; if the sensor output $\mathbf a$ were generated from signals $\mathbf s$ through a set of linear observations, $\mathbf s$ could be recovered with fidelity given fewer than $N$ measurements -- an otherwise underdetermined problem. %Specifically, if the activity is generated from the signal without the addition of noise, then the original signal $\mathbf s$ can be decoded by 
Specifically, this estimate is the vector which minimizes the $L_1$ norm of the signal subject to the measured activity:
\begin{align}
\hat {\mathbf s} = \arg \min \sum_i^N |s_i| \quad \text {s.t.  } \mathbf a = \mathbf M \mathbf s,
\label{eq:compressed_sensing_formulation}
\end{align}
where $\mathbf M$ is an $M$x$N$ measurement matrix.

But odor binding and transduction are nonlinear processes: olfactory encoding cannot be described by the simple linear input-output relation $a^\rho = M^{\rho i} s_i$, rather a more general formulation such as the steady state expression Eq.~\ref{eq:steady_state_activity}. One could in principle carry out the optimization in Eq.~\ref{eq:compressed_sensing_formulation} with the replacement of the linear constraint,  but the recovery of $\mathbf s$ is not straightforward as nonlinear programming contains no guarantees on convexity, and may well converge to a local minimum. To incorporate the odor binding dynamics in a sensible way, we assume the activity levels relax to their steady state values, Eq.~\ref{eq:steady_state_activity}, but use only a first-order approximation of the full binding and activation process for decoding. 

Adopting the viewpoint that the neural system can learn the mean background stimulus, we linearize fluctuations of the sparse odor signals around a known background $\bar{\mathbf s}$, estimating only the fluctuations $\Delta \mathbf s = \mathbf s - \bar{\mathbf s}$ via Eq.~\ref{eq:compressed_sensing_formulation}. In the limit $K_*^{\rho i} \ll s_i \ll K^{\rho i}$, this gives a measurement matrix $M^{\rho i} = (e^{\epsilon^\rho_{\text u}}/K_*^{\rho i})/(\sum_{i=1}^N s_i^0/K_*^{\rho i} + e^{\epsilon^\rho_{\text u}})^2$. 