%% model description in main text


\subsection*{Discrete-state stochastic olfactory receptor model}

We describe the sensory system by a stochastic discrete-state model, analogous to those used in describing ligand binding to chemoreceptor homodimers in bacterial chemotaxis. Deferring for now the effect of temporal fluctuations, we treat the dynamics in steady state, where the combination of odorant (un)binding and receptor (in)activation conspire to produce mean-field activity levels in response to given odorant stimuli. Here, we model the sensory complex as a single unit consisting of an olfactory receptor (OR) and and a co-receptor protein Orco, known to be expressed in the majority of insect ORNs, including all OR-expressing ORNs in \textit{drosophila}. This Or-Orco complex may exist in either an activated or inactivated state, and may be independently bound or unbound to volatile odorant molecular signals $s_i$. Binding obeys a stochastic jump process, where the affinity for binding is higher when activated. To simplify the analysis, we assume that a given receptor complex $R^\rho$ may bind at most a single monomolecular odorant at one time. Among a selection of $N$ such odorants, $R^{\rho}$ may therefore live in one of $N+1$ inactivated states with associated concentrations, $\{[R^\rho], [R^\rho s_1], [R^\rho s_2],...,[R^\rho  s_N]\}$ or $N+1$ activated states with concentrations $\{[R^\rho _*], [R^\rho _*s_1], [R^\rho _*s_2],...,[R^\rho _*s_N]\}$. 
Assuming that the binding of odors is faster than the time to activate or deactivate, the former is approximately quasistatic, with steady state probabilities
\begin{align}
p_{\text b}^{\rho i} = \frac{\frac{[s_i]}{K^{\rho i}}}{1 + \sum_i^N \frac{[s_i]}{K^{\rho i}}}, \qquad  
p_{\text b*}^{\rho i} = \frac{\frac{[s_i]}{K_*^{\rho i}}}{1 + \sum_i^N \frac{[s_i]}{K_*^{\rho i}}},
\label{eq:steady_state_binding_probabilities}
\end{align}
when (in)activated, and where $K^{\rho i}$ and $K^{\rho i}_*$ are the corresponding disassociation constants of the individual jump processes. Likewise, the activation kinetics when (un)bound obey Master equations
\begin{align}
\frac{d[R_*^\rho]}{dt} &= w_{\text u +}^\rho [R^\rho] - w_{\text u -}^\rho [R_*^\rho] \\
\frac{d[R_*^\rho s_i]}{dt} &= w_{\text b +}^{\rho i} [R^\rho s_i] - w_{\text b -}^{\rho i} [R_*^\rho s_i]
\label{eq:activation_master_equations}
\end{align}

When unbound, the free energy difference between inactivated and activated receptors is signal-independent, but may be distinct for differing receptors; we denote these as $\epsilon^\rho_{\text u}\equiv\ln(w^\rho_{\text u +}/w^\rho_{\text u -})$. Detailed balance enforced on 4-cycles involving a single ligand, therefore, demands 
$K_*^{\rho i}w_{\text b+}^\rho w_{\text u -}^\rho / 
K^{\rho i} w_{\text u +}^\rho w_{\text b -}^\rho \equiv 1$, 
whereby the free energy difference between activated and inactivated states when bound depends on the disassociation constants:
$\ln(w^{\rho i}_{\text b +}/w^{\rho i}_{\text b -}) \equiv
\epsilon_{\text b}^{\rho i} =
\epsilon_{\text u}^{\rho i} + \log(K_*^{\rho i}/K^{\rho i})$. 

In the mean-field limit, the collective action of binding and activation produces an activity level for each receptor: a sum of the concentrations of all activated states, bound or unbound. 
%$a^\rho = [R^\rho_*] + \sum_i^N[R^\rho_* s_i]$, 
The dynamics of these activity levels then obeys a linear rate equation, which in steady state relaxes to
\begin{align}
\bar a^\rho(\{[s_i]\}, \epsilon_{\text {u}}^\rho) = \left( 1 + e^{\epsilon^\rho_{\text u}}\frac{1 + \sum_{i=1}^N [s_i]/K^{\rho i}}{1 + \sum_{i=1}^N [s_i]/K_*^{\rho i}}\right)^{-1}.
\label{eq:steady_state_activity}
\end{align}
This expression is a simple generalization of the analogous expression for a 4-state receptor model with a single binding site. 
%In the limit that the system has a higher affinity for activation when bound than when unbound, we have $KK^{\rho i}_* \ll KK^{\rho i}$, and Eq.~\ref{eq:steady_state_activity} reduces to $\bar a^\rho = (1 + e^{\epsilon_{\text b}^\rho})

At a much larger timescale than odorant binding and receptor complex activation, the free energies $\epsilon^\rho_{\text u}$ may be modulated by adaptive feedback, adjusting in such a way to, ideally, maintain the sensitivity and range of the ORNs in response to fluctuating environments. We model this in a minimal fashion via $\dot \epsilon_{\text u}^\rho = \beta^\rho(a_0^\rho - \bar a^\rho)$, which assumes that the activity levels may decay at differing rates and to differing backgrounds for distinct receptor complexes $R^\rho$. {\color {blue} Say something about WL here?}


%$\dot a^\rho = (1 - a^\rho)w_+^\rho + w_-^\rho a$, where the transition rates $w^\rho_+$ and $w_-^\rho$ are appropriately weighted sums of $w_{\text u +}$, $w_{\text u -}$, $w_{\text u}$, and $w_{}
%$w_+ = (1-\sum_i^N p^{\rho i}_{\text b})w_{\text u +}^\rho +  \sum_i^N p^{\rho i}_{\text b} w_{\text b +}^{\rho i}$ and $w_- = (1-\sum_i^N p^{\rho i}_{\text b})w_{\text u -}^\rho +  \sum_i^N p^{\rho i}_{\text b} w_{\text b -}^{\rho i}$.
%The associated master equations for $2N$ possible 2-state jump processes are
%\begin{align}
%\frac{d [R^\rho s_i]}{dt} &= k_-^{\rho i} s_i [R^\rho] - k_+^{\rho i} [R^\rho s_i] \\
%\frac{d [R^\rho_* s_i]}{dt} &= k_-^{\rho i} s_i [R_*^\rho] - k_+^{\rho i} [R_*^\rho s_i] 
%\end{align} 
